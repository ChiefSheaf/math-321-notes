\begin{nquote}{: Dr. Joshua Zahl 03/27/2024}
    ``The bulk Fourier analysis is starting at a picture and figuring out what I what it tells you.''
\end{nquote}

\begin{ndef}{}
    The \(N^{\text{th}}\) partial Fourier series of \(f\) is 
    \begin{equation*} 
        S_N(x):=\sum_{n=-N}^N c_ne_n(x).
    \end{equation*}
\end{ndef}

\begin{nlemma}{}
    For \(x\in\R\), \(S_N(x)=\displaystyle\int_0^1 f(x-t)D_N(t) \, dt\), where 
    \begin{equation*} 
        D_N(t)=\sum_{n=-N}^N e^{2\pi int}=\frac{\sin(2\pi(N+\frac{1}{2})t)}{\sin(\pi t)}.
    \end{equation*}
    \(D_N\) is called the Dirichlet kernel.
\end{nlemma}
If we graph this function, it looks like a sine curve bound by an asymptotic envelope on all sides, and it oscillates very quickly (looking at the \href[]{https://en.wikipedia.org/wiki/Dirichlet_kernel}{Wikipedia page} is the best way of visualizing it). It oscillates very quickly.
\begin{proof}[Proof step-1]
    Note that 
    \begin{align*} 
        D_N(t)=&e^{-2\pi iNt}\sum_{k=0}^{2N} e^{2\pi int}\\
              =&e^{-2\pi iNt}\left(\frac{e^{2\pi i(2N+1)t}-1}{e^{2\pi it}-1}\right)\frac{e^{-\pi it}}{e^{-\pi it}}\\
              =&\frac{e^{2\pi i(N+\frac{1}{2})t}-e^{-2\pi i(N+\frac{1}{2})t}}{e^{\pi it}-e^{-\pi it}}\\
              =&\frac{\sin(2\pi(N+\frac{1}{2})t)}{\sin(\pi t)}.
    \end{align*}
\end{proof}
\begin{proof}[Proof step-2]
    Now,
    \begin{align*} 
        S_N(x)=&\sum_{x=-N}^N c_n e^{2\pi inx}\\
              =&\sum_{x=-N}^N \ip{f_n}{e_n}e^{2\pi inx}\\
              =&\sum_{x=-N}^N\int_0^1 f(t) e^{2\pi in(-t)}e^{2\pi inx} \, dt\\
              =&\sum_{x=-N}^N\int_0^1 f(t)e^{2\pi in(x-t)} \, dt\\
              =&\int_x^{x+1} f(x-s)D_N(s) \, ds \tag{\(\dagger\)}\label{1-periodic}\\
              =&\int_0^1 f(x-s) D_N(s) \, ds
    \end{align*}
    where \cref{1-periodic} is where we use the fact that the function is \(1-\)periodic.
\end{proof}
\begin{ndef}{: Lipschitz continuous at a point}
    Let \(f:\R\to\C\), \(x\in\R\). We say \(f\) is Lipschitz continuous at \(x\) if there exists \(\delta>0\) and \(L>0\) such that for all \(y\in R\) having \(|x-y|<\delta\), we have \(|f(x)-f(y)|\leq L|x-y|\).

    \medskip

    Alternatively for all \(t\in\R\) having \(|t|<\delta\), \(|f(x+t)-f(x)|\leq L|t|\).
\end{ndef}

\begin{ntheorem}{: Baby Rudin 8.14}
    Let \(f:\R\to\C\) be 1-periodic and integrable on \([0,1]\). Suppose \(f\) is Lipschitz continuous at \(x\). Then, \(\displaystyle\lim_{N\to\infty}S_N(x)=f(x)\).
\end{ntheorem}
Recall we have already shown this result if \(\{D_N(x)\}\) was an approximate identity. It is not one, but it has properties that resemble it. The proof should looks similar.
\begin{proof}
    To begin we compute a few things that will make our lives easier:
    \begin{enumerate}
        \item Note that 
        \begin{equation*} 
            \int_0^1 D_N(t) \, dt=\ip{D_N}{1}=\ip{D_N}{e_0(x)}=1.
        \end{equation*}
        
        \item Also,
        \begin{align*} 
            D_N(t)=&\frac{1}{\sin(\pi t)}[\sin(\pi t)\cos(2\pi Nt)+\sin(2\pi Nt)\cos(\pi t)]\\
            =&\cos(2\pi Nt)+\cot(\pi t)\sin(2\pi Nt).
        \end{align*}

        \item If \(g\) is 1-periodic and integrable on \([0,1]\),
        \begin{equation*} 
            \sum_{n=-\infty}^{\infty} |c_n|^2\leq \int_0^1 |g|^2 \, dx<\infty\implies c_n\to 0,
        \end{equation*}
        as \(n\to \infty\) or \(n\to-\infty\). This is called the \emph{Riemann-Lebesgue} lemma (or maybe a stronger version of this is called that.)

        \medskip
        
        As a consequence,
        \begin{equation*} 
            \lim_{N\to\infty}\ip{g}{\sin(2\pi Nx)}\to 0.
        \end{equation*}
    \end{enumerate}
    Therefore, 
    \begin{align*} 
        |\ip{g}{\sin(2\pi Nx)}|=&\left|\ip{g}{\sin(2\pi Nx)}\right|\\
                            \leq&\frac{1}{2}|\ip{g}{e^{2\pi inx}}|+\frac{1}{2}|\ip{g}{e^{-2\pi inx}}|\\
                               =&\frac{1}{2}(|c_n|+|c_n|)=|c_n|\to 0.
    \end{align*}
\end{proof}
The intuition behind is that Dirichlet kernel is oscillating very quickly; we say that it is oscillating in a mean zero way, meaning that it has mean zero almost everywhere.


