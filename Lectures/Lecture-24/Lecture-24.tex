\begin{nquote}{: Dr. Joshua Zahl 03/11/2024}
    ``What does obstruction mean? I looked it up in the dictionary before coming to class.'' - On the idea for Stone-\Weierstass having an obstruction.
\end{nquote}

\begin{ntheorem}{: Baby Rudin 7.29}
    Let \(\ms{A}\) be an algebra of bounded functions. Then, the uniform closure \(\Clu{\ms{A}}\)  is a uniformly closed algebra.
\end{ntheorem}
\begin{proof}
    Recall \((\mc{X},d)\) is a metric space of bounded functions \(\mc{E}\to\C\) (or \(\mc{E}\to\R\)), \(\Clu{\ms{A}}\) is the closure of \(\ms{A}\) in \((\mc{X},d)\), and \(\Clu(\ms{A})\) is obviously closed. Hence, we have shown that \(\Clu{\ms{A}}\) is uniformly closed. We now verify that this is an algebra.

    \medskip

    Let \(f,g\in\Clu(\ms{A})\), \(c\in\C\) (or \(c\in\R\)). Let \(\{f_n\}\subseteq\ms{A}\), such that \(f_n\to f\) uniformly, \(\{g_n\}\subseteq\ms{A}\), such that \(g_n\to g\) uniformly. All we need to show is that
    \begin{align*} 
        f_n+g_n\to&f+g~\text{uniformly}\\
        f_n\cdot g_n\to&f\cdot g~\text{uniformly}\\
        cf_n\to&cf~\text{uniformly}.
    \end{align*}
    These are straightforward and left as an exercise. Note that your proof here should use the fact that these functions are uniformly bounded, because otherwise we can run into trouble; consider the following example: For \(\mc{E}=\R\), \(f_n(x)=x\), \(g_n(x)=\displaystyle\frac{1}{n}\), we have \(f_n(x)g_n(x)=\displaystyle\frac{x}{n}\).
\end{proof}

\subsection{Idea for Stone-\Weierstass}
Let \(\mc{K}\) be a compact set, \(\ms{A}\) an (real) algebra of continuous function \(f:\mc{K}\to\R\). Then, \(\Clu(\ms{A})\) is the algebra of \emph{all} continuous functions \(f:\mc{K}\to\R\). This is true \emph{unless} there is an \emph{obvious} obstruction; spoiler, there is. We will now acknowledge and destroy these obstructions.

\begin{ndef}{: Separates points}
    Let \(\mc{E}\) be a set, \(\ms{A}\) a set of functions \(f:\mc{E}\to\C\) (or \(\mc{E}\to\R\)). We say that \(\ms{A}\) \emph{\textbf{Separates points}} if for all \(x,y\in\mc{E}\), there exists \(f\in\ms{A}\) such that \(f(x)\neq f(y)\).
\end{ndef}

\begin{example}
    Connsider \(\mc{E}=[-1,1]\) or \(\mc{E}=\R\). If \(\ms{A}\) is all polynomial functions, then we are fine, but if \(\ms{A}\) is the set of all even polynomial functions, then this does not separate points, since \(f(x)=f(-x)\) for all \(x\in\mc{E}\).
\end{example}
Failure to separate points is an obstruction to Stone-\Weierstass.

\begin{ndef}{: Vanishes at no point of \(\mc{E}\)}
    Let \(\mc{E}\) and \(\ms{A}\) be as before. We say that \(\ms{A}\) \emph{\textbf{vanishes at no point of \(\mc{E}\)}} if for each \(x\in\mc{E}\), there is a function \(f\in\ms{A}\) such that \(f(x)\neq 0\).
\end{ndef}
\begin{example}
    The previous example \(\mc{E}=[-1,1]\) or \(\mc{E}=\R\), \(\ms{A}\) is all polynomial functions; this works just fine once again. However, now if \(\ms{A}\) is the set of all odd polynomial functions, then we infer from \(f(-x)=-f(x)\) for all \(x\in\mc{E}\), that \(f(0)=0\). So this does not satisfy the definition.
\end{example}
Vanishing at a point of \(\mc{E}\) is an obstruction.
\begin{digression}
    If \(\ms{A}\) is the set of all odd polynomials, it is clearly not an algebra, so what is the smallest algebra that contains all odd polynomials? This is the ideal generated by \(x\), \((x)\) over the ring of polynomials \(\C[x]\) (or \(\R[x]\)). A natural question to ask ourselves now in this context is what is the uniform closure of this? This is not something we have answered so far, but is something worth thinking about.
\end{digression}

Having acknowledged these obstructions, we can now theorise:

\begin{ntheorem}{: Stone-\Weierstass theorem for reals (Baby Rudin 7.32)}
    Let \(\mc{K}\) be a compact metric space, \(\ms{A}\) an algebra of continuous functions \(f:\mc{E}\to\R\). Suppose that \(\ms{A}\) separates points and vanishes at no point of \(\mc{K}\). Then, the uniform closure \(\Clu(\ms{A})\) is the algebra of all continuous functions \(f:\mc{K}\to\R\).
\end{ntheorem}
The proof has 3 main steps, that we will deal with as lemmas.
\begin{lemma}
    Let \(\mc{K}\) and \(\mc{A}\) as above. If \(f\in\ms{A}\), then \(|f|\in\Clu(\ms{A})\).
\end{lemma}
\begin{proof}
    Let \(f\in\ms{A}\), \(M:=\sup\{|f(x)|\st x\in\mc{K}\}\), \(M<\infty\). Let 
    \(\eps>0\); we wish to find \(g\in\ms{A}\) such that 
    \begin{equation*} 
        \sup_{x\in\mc{K}}\big||f(x)|-g(x)\big|<\eps,
    \end{equation*}
    which would tell us \(|f|\in\ms{A}\).

    \medskip

    By the \Weierstass approximation theorem (or an explicit computation), there exists a polynomial function \(q\) such that \(|q(y)-|y||<\displaystyle\frac{\eps}{2}\) for all \(y\in[-M,M]\).

    \smallskip

    Let \(p(y)=q(y)-q(0)\implies p(0)=0\). Hence,
    \begin{align*} 
        \big|p(y)-|y|\big|=&\big|q(y)-q(0)-|y|-0\big|\\
                       \leq&\big|q(y)-|y|\big|+\big|q(0)-0\big|<\frac{\eps}{2}+\frac{\eps}{2}=\eps,\quad\text{for all}~y\in[-M,M].
    \end{align*}        
    Write \(p(x)=\displaystyle\sum_{k=1}^n c_kx^k\), and since \(\ms{A}\) is an algebra, we define 
    \begin{equation*} 
        p(f):=\underbrace{\sum_{k=1}^n c_kf^k}_{g}\in\ms{A}.
    \end{equation*}
    For \(x\in\mc{K}\), 
    \begin{align*} 
        \big|g(x)-|f(x)|\big|=&\big|p(f)(x)-|f(x)|\big|\\
                             =&\big|p(f(x))-|f(x)|\big|<\eps.
    \end{align*}
\end{proof}
\begin{note}[Usage of \Weierstass approximation in the proof]
    Recall that we said that Stone-\Weierstass would imply the \Weierstass approximation, but then using the approximation theorem in the proof seems a bit circular. It is not, because we are using the approximation theorem to approximate a very specific function by a polynomial function, which can also be achieved by an explicity computation; we are just making our life easier.
\end{note}
