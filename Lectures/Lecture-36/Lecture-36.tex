\begin{nquote}{: Dr. Joshua Zahl 04/12/2024}
    ``What you're describing sounds like we have to think, which we'd like to avoid." -- on a student's answer to his question.

    \medskip

    ``Maybe it's bad, I seem to be discouraging people from thinking." -- after something completely different about 15 minutes later.
\end{nquote}

\subsection{Equidistribution}

\begin{ndef}{: Fractional part of a real number}
    If \(x\in\R\), \(\expec{x}=x-\floor{x}\in[0,1)\), i.e., \(\expec{x}=x(\mod 1)\). This is called the \emph{\textbf{fractional part}} of a the real number \(x\).
\end{ndef}

Let \(\alpha\in\R\); consider the sequence \(\expec{n\alpha}\).
\begin{itemize}
    \item If \(\alpha=\displaystyle\frac{p}{q}\in\Q\), \(\expec{n\alpha}\) is periodic.
    
    \item If \(\alpha\notin\Q\), every value of \(\expec{n\alpha}\) is distinct. If \(\expec{n\alpha}=\expec{m\alpha}\), then \(n\alpha-m\alpha\in\Z\), so \(\alpha\in\Q\).
\end{itemize}

\begin{ntheorem}{ (Kronecker): E}
    If \(\alpha\notin\Q\), then \(\expec{n\alpha}\) is \emph{dense} in \([0,1)\).
\end{ntheorem}

\begin{ndef}{: Equidistributed}
    A sequence \((x_n)\subseteq [0,1)\) is called \emph{\textbf{equidistributed}} if: for every interval \(\mc{I}\subseteq[0,1)\), 
    \begin{equation*} 
        \lim_{N\to\infty}\frac{\#\{n=1,\dots, N\st x_n\in\mc{I}\}}{N}\to\ell(\mc{I}),
    \end{equation*}
    where \(\ell(\mc{I})\) is the length of \(\mc{I}\).
\end{ndef}

\begin{ntheorem}{ (Weyl): F}
    If \(\alpha\notin\Q\), then \(\expec{n\alpha}\) is equidistributed.
\end{ntheorem}
\begin{note}
    Theorem F implies theorem E, since being equidistributed is a stronger notion of being dense.
\end{note}
In fact, there is something else worth noting here. If \(f:\R\to\R\) is 1-periodic and integrable on \([0,1]\) and \(\alpha\notin\Q\), then 
    \begin{equation*} 
        \lim_{N\to\infty}\frac{1}{N}\sum_{n=1}^N f\big(n\alpha\big)=\int_0^1 f(x) \, dx.\tag{\(\diamondsuit\)}\label{implies Weyl}
    \end{equation*}
\begin{claim}
    Theorem F is implied by \cref{implies Weyl}.
\end{claim}
\begin{proof}
    \emph{Step-1:} We verify that \(\cref{implies Weyl}\) is true when \(f(x)=e^{2\pi ikx}\), \(k\in\Z\).

    \medskip

    If \(k=0\), \(f(x)=1\): \(1=1\) is tautology.

    \medskip

    For \(k>0\), RHS of \cref{implies Weyl} is zero by a straightforward computation. For LHS:
    \begin{align*} 
        \frac{1}{N}\sum_{n=1}^N f(n\alpha)=&\frac{1}{N}\sum_{n=1}^N e^{2\pi ikn\alpha}\\
        =&\frac{1}{N}e^{2\pi ik\alpha}\left(\frac{1-e^{2\pi ikN\alpha}}{1-e^{2\pi ik\alpha}}\right)\to 0,~\text{as}~N\to\infty.
    \end{align*}

    \medskip

    \emph{Step-2:} Note that if \(\cref{implies Weyl}\) is true for \(f,g\), then it is true for \(af+g\) where \(a\in\R\). Hence, by limit laws and linearity of the integral, we conclude that \cref{implies Weyl} holds for trigonometric polynomial functions.

    \medskip

    \emph{Step-3:} We can further extend this to continuous 1-periodic functions using the \(\eps/3\) argument using Step-1, and the fact that continuous functions can be uniformly approximated using trigonometric polynomial functions.

    \medskip

    \emph{Step-4:} Finally, we extend this to a function that is 1-periodic and integrable on one period. We know that we can approximate any Riemann integrable functions using continuous functions -- using step function approximations as an intermediate step -- in the \(L^1\) sense. But here, we wish to have better control of the approximators.

    \medskip

    Given \(\eps>0\), let \(f_+, f_-\) be continuous, 1-periodic functions with \(f_-(x)\leq f(x)\leq f_+(x)\), and 
    \begin{equation*} 
        \int_0^1 \big(f_+(x)-f_-(x)\big) \, dx<\eps.
    \end{equation*}
    We let \(A:=\displaystyle\int_0^1 f_-(x) \, dx\) and \(B:=\displaystyle\int_0^1 f_+(x) \, dx\), and note that we can split this integral using Baby Rudin theorem 6.12. We don't know if the limit of their corresponding sums in \cref{implies Weyl} exists, but we know that the limsup and liminf exist; they might exist as extended reals, but that is fine by us. Hence, note that
    \begin{align*} 
        \limsup_{N\to\infty}\frac{1}{N}\sum_{n=1}^N f(n\alpha)\leq&\limsup_{N\to\infty}\frac{1}{N}\sum_{n=1}^Nf_+(n\alpha)\\
        =&\lim_{N\to\infty}\frac{1}{N}\sum_{n=1}^Nf_+(n\alpha),
    \end{align*}
    and,
    \begin{equation*} 
        \liminf_{N\to\infty}\frac{1}{N}\sum_{n=1}^N f(n\alpha)\geq\lim_{N\to\infty}\frac{1}{N}\sum_{n=1}^N f_-(n\alpha.)
    \end{equation*}
    Therefore, 
    \begin{equation*} 
        \limsup_{N\to\infty}\frac{1}{N}\sum_{n=1}^N f(n\alpha)=\liminf_{N\to\infty}\frac{1}{N}\sum_{n=1}^N f(n\alpha)=\int_0^1 f(x) \, dx.
    \end{equation*}
\end{proof}

\begin{center}
    \textbf{End of MATH 321 :-)}
\end{center}

\begin{comment}
    \begin{proof}[``Proof'' by example]
    We don't prove this here, but show an example. If \(\mc{I}\subseteq[0,1)\) is an interval, let \(f\) be periodic \(\mc{I}\) and \(f(x)=\chi_{\mc{I}}(x)\) on \([0,1)\). If \(\mc{I}=\displaystyle\left(\frac{1}{2},\frac{3}{4}\right)\), then it can be inferred after some work that \cref{implies Weyl} implies Theorem F.
\end{proof}
\end{comment}


