\begin{nquote}{: Dr. Joshua Zahl 01/19/2024}
	No quotes today :(
\end{nquote}

We will now prove theorem 6.10:
\begin{proof}
	Let \(N:=\displaystyle\sup_{x\in[a,b]}{|f|}\); this is finite since \(f\) is bounded. Let \(\mc{E}:=\{e_1,\dots,e_k\}\) be the set of points where \(f\) is discontinuous.
	
	\medskip
	
	Let \(\eps_1>0\). Since \(\alpha\) is continuous at each \(e_i\in\mc{E}\), we can pick \(u_i<e_i<v_i\), where \(u_i,v_i\in[a,b]\), such that \(0\leq\alpha(v_i)-\alpha(u_i)<\eps_1\). The inequalities can be equality if \(e_i=a\) or \(e_i=b\).
	
	\medskip
	
	Let \(\mc{K}:=[a,b]\backslash\displaystyle\bigcup_{i=1}^k (u_i,v_i)\). Since \(\mc{K}\) is closed and bounded, it is compact. Furthermore, since \(f\) is continuous on \(\mc{K}\), it is uniformly continuous on \(\mc{K}\): for all \(x,y\in\mc{K}\) with \(|x-y|<\delta\), we have \(|f(x)-f(y)|<\eps_1\).
	
	\medskip
	
	Let \(\{y_i\}\subseteq\mc{K}\) be a set of points such that for every \(x\in\mc{K}\), there is an index \(i\) such that \(y_i\leq x\leq y_{i+1}\), and \(0<y_{i+1}-y_i<\delta\). Also, let \(\mc{P}:=\displaystyle\{u_i,v_i\}_{i=1}^k\cup \{y_i\}\cup\{a,b\}\) (might have to re-order to pout these in increasing order). Hence, 
	\begin{align*}
		0\leq U(\mc{P},f,\alpha)-L(\mc{P},f,\alpha)=&\sum_{i=1}^n(M_i-m_i)\Delta\alpha_i~(\text{for interval}~[x_{i-1},x_i])\\
		&\leq \underbrace{k(2N)\eps_1}_{[u_i,v_i]~\text{intervals}}+\underbrace{\eps_1\left(\alpha(b)-\alpha(a)\right)}_{[y_{i-1},y_i]~\text{intervals}}.
	\end{align*}
	Given \(\eps>0\), choose \(\eps_1\) such that 
	\begin{equation*}
		k(2N)\eps_1+\eps_1\left(\alpha(b)-\alpha(a)\right)<\eps;
	\end{equation*}
	we use the partition \(\mc{P}\). Therefore, we have shown that \(f\in\mc{R}_{\alpha}[a,b]\).
\end{proof}

\begin{fft}
	What if \(f\) and \(\alpha\) are both discontinuous at a common point? If \(f\in\mc{R}_{\alpha}[a,b]\) always? Does it depend on \(f\) and \(\alpha\)? Or is this never true? 
\end{fft}
\begin{proof}[Solution]
	Consider the case
	\begin{equation*}
		f(x)=\alpha(x)=\begin{cases}
						0&x<0\\
						1&x\geq 0
					   \end{cases}.
	\end{equation*}
	Let \(\mc{P}=\{-1=x_0,x_1,x_2,\dots,x_n=1\}\) be the partition on the interval \([-1,1]\). There are two cases that we need to consider here: if we look at 
	\begin{equation*}
		U(\mc{P},f,\alpha)-L(\mc{P},f,\alpha)=\sum_{i=1}^n(M_i-m_i)\Delta\alpha_i,
	\end{equation*}
	the only interesting term is the one about the point of discontinuity (the origin); we can choose our partition to be such that the origin in between two points of the partition, but if we work this out, we get that \(M_k-m_k=1-0=1\), and \(\alpha_{k}-\alpha_{k-1}=1\), so \(f\notin\mc{R}_{\alpha}[a,b]\). In this case that zero is one of the partition points, we end up getting the same thing, so this function turns out to not be Riemann-Stieltjes integrable with this integrator. 
	
	\medskip
	
	However, while keeping \(f\) the same, if we slightly change \(\alpha\) to be 
	\begin{equation*}
		\alpha(x)=\begin{cases}
					0&x\leq 0\\
					1&x>0
				  \end{cases},
	\end{equation*}
	and we let the origin be one of the partition points, we see that over the interval \([s,0]\), where \(s\in\mc{P}\), 
	\begin{equation*}
		U(\mc{P},f,\alpha)-L(\mc{P},f,\alpha)=\sum_{i=1}^n(M_i-m_i)\Delta\alpha_i=(1-0)(0-0)=0,
	\end{equation*}
	so this in fact is now Riemann-Stieltjes integrable. It is still Riemann-Stieltjes integrable if we consider an interval of the form \([0,t]\) for \(t\in\mc{P}\).
	
	\medskip
	
	This is particularly interesting because if we compute the integral \(\displaystyle\int_{-1}^1 f \, d\alpha\), we get that it evaluates to zero, which means in this case, even though it is integrable, the integrator was unable to detect the step up in the function. So we conclude that if the function and the integrator share a point of discontinuity, then sometimes the function is still Riemann-Stieltjes integrable. However, funny things happen in such situations.
\end{proof}

\begin{figure}[H]
	\centering
	\begin{tikzpicture}[scale=1.3]
		% the function f
		
		\draw[-] (-5,1)--(5,1);
		
		\node[left] at (-5,1) {\(f:\)};
		
		\node[] at (-5,1) {\([\)};
		
		\node[] at (5,1) {\(]\)};
		
		\draw[fill] (-2.5,2.5) circle (1.5pt);
		
		\draw[] (-2.5,1.9) circle (1.5pt);
		
		\draw[fill] (-2.5,1) circle (1pt);
		
		\node[below] at (-2.5,1) {\scalebox{0.85}{\(e_1\)}};
		
		\draw[-, dashed] (-2.5,3)--(-2.5,-2);
		
		\draw[] (-4.5,1.2) to [curve through = {(-4,1.5)..(-3.5,1.9)..(-3,1.6)..(-2.5,1.9)..(-2.3,2)..(-2,1.7)..(-1.7,1.5)}] (-1.5,1.6);
		
		\draw[] (-1.5,1.6) circle (1.5pt);
		
		\draw[fill] (-1.5,2) circle (1.5pt);
		
		\draw[-, dashed] (-1.3,1)--(-1.3,-2);
		
		\draw[-, dashed] (-1.7,1)--(-1.7,-2);
		
		\node[] at (-1.7,1) {\([\)};
		
		\node[] at (-1.3,1) {\(]\)};
		
		\node[above] at (-1.7,1.1) {\scalebox{0.85}{\(u_2\)}};
		
		\node[above] at (-1.3,1.1) {\scalebox{0.85}{\(v_2\)}};
		
		\draw[fill] (-1.5,1) circle (1pt);
		
		\node[below] at (-1.5,1) {\scalebox{0.85}{\(e_2\)}};
		
		\draw[] (-1.5,2) to [curve through = {(-1,2.2)..(-0.5,2)..(0,1.9)}] (0.1,1.9);
		
		\draw[] (0.1,1.9) circle (1.5pt);
		
		\draw[-, dashed] (0.1,3)--(0.1,-2);
		
		\draw[fill] (0.1,1) circle (1pt);
		
		\node[below] at (0.1,1) {\scalebox{0.85}{\(e_3\)}};
		
		\node[below] at (-5,0.9) {\(a\)};
		
		\node[below] at (5,0.9) {\(b\)};
		
		\draw[fill] (0.1,1.3) circle (1.5pt);
		
		\draw[] (0.1,1.3) to [curve through = {(0.3,1.5)..(0.5,1.47)..(1,1.45)..(1.5,1.5)..(2,1.75)..(2.5,2)..(3,2.2)..(3.5,1.8)..(4,1.65)..(4.5,1.5)}] (5,1.4);
		
		\draw[] (2.5,2) circle (1.5pt);
		
		\draw[fill] (2.5,1.5) circle (1.5pt);
		
		\draw[-, dashed] (2.5,3)--(2.5,-2);
		
		\node[below] at (2.5,1) {\scalebox{0.85}{\(e_4\)}};
		
		
		% the integrator alpha
		
		\draw[-] (-5,-2)--(5,-2);
		
		\node[] at (-5,-2) {\([\)};
		
		\node[] at (5,-2) {\(]\)};
		
		\node[left] at (-5,-2) {\(\alpha:\)};
		
		\node[] at (-2.7,-2) {\((\)};
		
		\node[] at (-2.3,-2) {\()\)};
		
		\node[below] at (-2.7,-2.1) {\scalebox{0.85}{\(u_1\)}};
		
		\node[below] at (-2.3,-2.1) {\scalebox{0.85}{\(v_1\)}};
		
		\node[] at (-1.7,-2) {\((\)};
		
		\node[] at (-1.3,-2) {\()\)};
		
		\node[below] at (-1.7,-2.1) {\scalebox{0.85}{\(u_2\)}};
		
		\node[below] at (-1.3,-2.1) {\scalebox{0.85}{\(v_2\)}};
		
		\node[] at (-0.1,-2) {\((\)};
		
		\node[] at (0.3,-2) {\()\)};
		
		\node[below] at (-0.1,-2.1) {\scalebox{0.85}{\(u_3\)}};
		
		\node[below] at (0.3,-2.1) {\scalebox{0.85}{\(v_3\)}};
		
		\node[] at (2.3,-2) {\((\)};
		
		\node[] at (2.7,-2) {\()\)};
		
		\node[below] at (2.3,-2.1) {\scalebox{0.85}{\(u_4\)}};
		
		\node[below] at (2.7,-2.1) {\scalebox{0.85}{\(v_4\)}};
		
		\node[] at (-4.8,-2) {\(\mid\)};
		
		\node[] at (-4.4,-2) {\(\mid\)};
		
		\node[] at (-4,-2) {\(\mid\)};
		
		\node[] at (-3.6,-2) {\(\mid\)};
		
		\node[] at (-3.2,-2) {\(\mid\)};
		
		\node[] at (-2.8,-2) {\(\mid\)};
		
		\node[] at (-2.4,-2) {\(\mid\)};
		
		\node[] at (-2,-2) {\(\mid\)};
		
		\node[] at (-1.6,-2) {\(\mid\)};
		
		\node[] at (-1.2,-2) {\(\mid\)};
		
		\node[] at (-0.8,-2) {\(\mid\)};
		
		\node[] at (-0.4,-2) {\(\mid\)};
		
		\node[] at (0,-2) {\(\mid\)};
		
		\node[] at (0.4,-2) {\(\mid\)};
		
		\node[] at (0.8,-2) {\(\mid\)};
		
		\node[] at (1.2,-2) {\(\mid\)};
		
		\node[] at (1.6,-2) {\(\mid\)};
		
		\node[] at (2,-2) {\(\mid\)};
		
		\node[] at (2.4,-2) {\(\mid\)};
		
		\node[] at (2.8,-2) {\(\mid\)};
		
		\node[] at (3.2,-2) {\(\mid\)};
		
		\node[] at (3.6,-2) {\(\mid\)};
		
		\node[] at (4,-2) {\(\mid\)};
		
		\node[] at (4.4,-2) {\(\mid\)};
		
		\node[] at (4.8,-2) {\(\mid\)};
		
		\draw[decorate,decoration={brace,amplitude=5pt,mirror,raise=1.5ex}]
		(4,-2) -- (4.4,-2) node[midway,yshift=-1.74em]{\scalebox{0.85}{\(\delta/2\)}};
		
		\draw[] (-5,-1.8) to [curve through = {(-4.7,-1.7)..(-4.4,-1.6)..(-3.8,-1.5)..(-3.4,-1.4)..(-2.6,-1.3)..(-2.2,-1.2)..(-1.7,-1.1)..(-1.3,-1)..(-0.8,-0.9)..(-0.4,-0.8)}] (0.8,-0.65);
		
		\draw[] (0.8,-0.65) circle (1.5pt);
		
		\draw[fill] (0.8,-0.45) circle (1.5pt);
		
		\draw[] (0.8,-0.45) to [curve through = {(1.4,-0.35)..(2,-0.3)..(3.5,-0.25)..(4,-0.2)}] (5,-0.1);
	\end{tikzpicture}
	\caption{``Proof by picture" for the theorem.}
\end{figure}

\begin{ntheorem}{: Baby Rudin 6.11}
	Let \(f:[a,b]\to\R\) be bounded, and \(\alpha:[a,b]\to\R\) be monotone increasing. Suppose \(f\in\mc{R}_{\alpha}[a,b]\). Suppose \(m\leq f(x)\leq M\) for all \(x\in[a,b]\). Let \(\pfi:[m,M]\to\R\) be continuous; then \(\pfi\circ f\in\mc{R}_{\alpha}[a,b]\).
\end{ntheorem}