\begin{nquote}{: Dr. Joshua Zahl 02/08/2024}
	No quotes today :(
\end{nquote}

\begin{ntheorem}{: Baby Rudin 7.11}
	Let \((\mc{M}_1,d_1)\) and \((\mc{M}_2,d_2)\) be metric spaces with \((\mc{M}_2,d_2)\) complete, i.e., \(\R\) or \(\C\). Let \(\mc{E}\subseteq \mc{M}_1\), and let \(\{f_n\}\) be a sequence of functions \(f_n:\mc{E}\to\mc{M}_2\), and suppose \(f_n\to f\) uniformly on \(\mc{E}\). 
	
	\medskip
	
	Let \(x\in\mc{M}_1\) be a limit point of \(\mc{E}\). Suppose \(\displaystyle\lim_{t\to x}f_n(t)=y_n\) exists for each \(n\); \(\{y_n\}\) is a convergent sequence, i.e., \(y_n\to y\in\mc{M}_2\), and \(\displaystyle\lim_{t\to x}f(t)=y\), i.e., 
	\begin{equation*}
		\lim_{t\to x}\underbrace{\lim_{n\to\infty}f_n(t)}_{f(t)}=\lim_{n\to\infty}\underbrace{\lim_{t\to x}f_n(t)}_{y_n}
	\end{equation*}
\end{ntheorem}
\begin{proof}
	\textbf{Step-1:} Show that \(\{y_n\}\) converges.
	
	\medskip
	
	It suffices to show that \(\{y_n\}\) is Cauchy. Let \(\eps>0\) be given. Choose \(N\) such that for all \(m,n>N\), for all \(t\in \mc{E}\), \(d_2(f_n(t),f_m(t))<\displaystyle\frac{\eps}{3}\), and thus
	\begin{align*}
		d_2(y_n,y_m)\leq&d_2(y_n,f_n(t))+d_2(f_n(t),y_m)\\
					\leq&d_2(y_n,f_n(t))+d_2(f_n(t),f_m(t))+d_2(f_m(t),y_m).
	\end{align*}
	We can choose \(t\) such that the above is at most \(<\displaystyle\frac{\eps}{3}+\frac{\eps}{3}+\frac{\eps}{3}=\eps\); call this the ``\(\displaystyle\frac{\eps}{3}\) trick".
	
	\medskip
	
	In conclusion, for all \(\eps>0\), there exists \(N\in\N\) such that for all \(m,n>N\), \(d_2(y_n,y_m)<\eps\), i.e., \(\{y_n\}\) is Cauchy, and by completeness of \((\mc{M}_2, d_2)\), hence convergent.
	
	\bigskip
	
	\textbf{Step-2:} Prove that \(f(t)\to y\) as \(t\to x\).
	
	\medskip
	
	For all \(t\in\mc{E}\) and \(n\),
	\begin{equation}
		d_2(f(t),y)\leq d_2(f(t),f_n(t))+d_2(f_n(t),y_n)+d_2(y_n,y). \tag{\(\star\)} \label{eps/3 for step 2}
	\end{equation}
	Let \(\eps>0\); since \(f_n\to f\) uniformly, there exists \(N_1\in\N\) such that for all \(n>N_1\), for all \(t\in\mc{E}\),
	\begin{equation*}
		d_2(f(t), f_n(t))<\frac{\eps}{3}.
	\end{equation*}
	Since \(y_n\to y\), there exists \(N_2\in\N\) such that for all \(n>N_2\), \(d_2(y_n,y)<\displaystyle\frac{\eps}{3}\). Let \(N:=\operatorname{max}\{N_1,N_2\}\). Applying \cref{eps/3 for step 2} with this choice of \(N\), we have 
	\begin{equation*}
		d_2(f(t),y)\leq \frac{\eps}{3}+d_2(f_N(t),y_N)+\frac{\eps}{3}.
	\end{equation*}
	Since \(\displaystyle\lim_{t\to x}f_N(t)=y_N\), there exists \(\delta>0\) such that for all \(t\in\mc{E}\), \(d_1(t,x)<\delta\), we have \(d_2(f_N(t),y_N)<\displaystyle\frac{\eps}{3}\). Hence, for all \(t\in\mc{E}\), for all \(x\) obeying \(d_1(t,x)<\delta\), we have 
	\begin{equation*}
		d_2(f(t),y)<\eps.
	\end{equation*}
\end{proof}

\begin{ncorollary}{: Baby Rudin 7.12}
	Let \((\mc{M}_1,d_1)\), \((\mc{M}_2,d_2)\), \(\{f_n\}\), \(f\), and \(\mc{E}\) be as before. If each \(f_n\) is continuous on \(\mc{E}\), and \(f_n\to f\) uniformly, then \(f\) is continuous on \(\mc{E}\).
	
	\medskip
	
	Effectively, ``the uniform limit of continuous functions is continuous."
\end{ncorollary}
\begin{proof}
	\(f\) is always continuous at isolated points, so we only need to consider limit points, \(x\in\mc{E}\cap\mc{E}'\), For every such \(x\), theorem 7.11 implies 
	\begin{equation*}
		f(x)=\lim_{n\to\infty}f_n(x)=\lim_{n\to\infty}\lim_{t\to x}f_n(t)=\lim_{t\to x}\lim_{n\to\infty}f_n(t)=\lim_{t\to x}f(t).
	\end{equation*}
\end{proof}

\subsection{Series of functions}
\begin{ndef}{: Convergence of a series of functions to a function}
	Let \(\mc{E}\) be a set, let \(\{f_n\}\) be a sequence of functions, \(f_n:\mc{E}\to\R\) or \(\mc{E}\to\C\), and let \(g:\mc{E}\to\R\) or \(\mc{E}\to\C\). We say \(\displaystyle\sum_{n\in\N} f_n\) converges point-wise (uniformly) to \(g\) is the sequence \(S_n:=\sum_{i=1}^{n}f_i\) converges point-wise (uniformly) to \(g\).
\end{ndef}
\begin{example}
	The series \(1+\displaystyle\sum_{n\in\N}\frac{x^n}{n!}\) converges to \(g(x)=e^x\)
	\begin{itemize}
		\item point-wise on \(\R\).
		
		\item uniformly on any bounded set \(\mc{E}\subseteq\R\), or any compact set \(\mc{K}\subseteq\R\).
	\end{itemize}
\end{example}

\begin{ntheorem}{: Weierstra\ss \(~M\)-test}
	Let \(\mc{E}\) be a set, \(f_n:\mc{E}\to\R\) or \(\mc{E}\to\C\). If \(|f_n(x)|\leq M\) for all \(n>N_0\in\N\), for all \(x\in\mc{E}\), and if \(\displaystyle\sum_{n=N_0}^{\infty}M_n<\infty\), then \(\displaystyle\sum_{n\in\N}f_n\) converges uniformly.
\end{ntheorem}