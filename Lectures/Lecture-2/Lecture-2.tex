\begin{nquote}{: 01/10/2024}
	No quotes today :(
\end{nquote}

\begin{figure}[H]
	\centering
	\begin{tikzpicture}
		\draw[Latex-Latex] (-4,0)--(4,0);
		
		\node[] at (-2.5,0) {\([\)};
		
		\node[] at (2.5,0) {\(]\)};
		
		\node[below] at (-2.5,0) {\(a\)};
		
		\node[below] at (2.5,0) {\(b\)};
		
		\draw[fill] (-1.25,0) circle (1pt);
		
		\draw[fill] (1.25,0) circle (1pt);
		
		\node[below] at (-1.25,0) {\(x_0\)};
		
		\node[below] at (1.25,0) {\(x\)};
		
		\draw[fill] (0,0) circle (1pt);
		
		\node[above] at (0,0) {\(c\)};
	\end{tikzpicture}
	\caption{Visualization of points in Taylor's theorem.}
\end{figure}

\begin{proof}
	We start by noting that for \(n=0\), \(\cref{taylor}\) says \(f(x)=f(x_0)+f'(c)(x-x_0)\). 
	
	\medskip
	
	Define \(A\in\R\) by 
	\begin{equation*}
		f(x)-P_n(x)=\frac{A}{(n+1)!}(x-x_0)^{n+1}.
	\end{equation*}
	Our goal here is to show that there exists a \(c\) between \(x_0\) and \(x\) such that \(f^{(n+1)}(c)=A\).
	
	\medskip
	
	Define \(g(t)=f(t)-P_n(t)-\displaystyle\frac{A}{(n+1)!}(t-x_0)^{n+1}\).
	
	\begin{figure}[H]
		\centering
		\begin{tikzpicture}
			\draw[Latex-Latex] (-5,0)--(5,0);
			
			\node[] at (-3.5,0) {\(\big[\)};
			
			\node[] at (3.5,0) {\(\big]\)};
			
			\node[below] at (-3.5,-0.15) {\(a\)};
			
			\node[below] at (3.5,-0.15) {\(b\)};
			
			\node[] at (-2.5,0) {\([\)};
			
			\node[] at (2.5,0) {\(]\)};
			
			\node[below] at (-2.5,-0.15) {\(a'\)};
			
			\node[below] at (2.5,-0.15) {\(b'\)};
			
			\draw[] (-3.5,0) to [curve through = {(-2.5,0.8)..(0,2)..(2.5,0.8)..(3,0)}] (3.5,-0.8);
			
			\node[] at (-2.5,0.8) {\(\mid\)};
			
			\node[] at (2.5, 0.8) {\(\mid\)};
 			
			\begin{comment}
				\draw[fill] (-1.25,0) circle (1pt);
				
				\draw[fill] (1.25,0) circle (1pt);
				
				\node[below] at (-1.25,0) {\(x_0\)};
				
				\node[below] at (1.25,0) {\(x\)};
				
				\draw[fill] (0,0) circle (1pt);
				
				\node[above] at (0,0) {\(c\)};
			\end{comment}
		\end{tikzpicture}
		\caption{Visualization of how we shrink the interval to possibly apply Rolle's theorem.}
	\end{figure}
	
	Observe
	\begin{align*}
		g(x_0)=&f(x_0)-P_n(x_0)\\
			  =&f(x_0)-f(x_0)-0\\
			  =&0,
	\end{align*}
	so \(g(x)=0\) by definition of \(A\). Hence, for \(j=0,\dots, n\), 
	\begin{align*}
		g^{(j)}(x_0)=&f^{(j)}(x_0)-P_n^{(j)}(x_0)-\frac{d^j}{dt^j}\left\{\frac{A}{(n+1)!}(t-x_0)^{n+1}\right\}\bigg|_{t=x_0}\\
					=&f^{(j)}(x_0)-f^{(j)}(x_0)-0\\
					=&0,
	\end{align*}
	which tells us that \(g^{(n+1)}(t)=f^{(n+1)}(t)-0-A\). Now, our goal is to find a \(c\) such that \(g^{(n+1)}(c)=0\).
	
	\medskip
	
	Note that 
	\begin{align*}
		&g(x_0)=0,g(x)=0~\text{by Rolle's thoerem, there exists}~c_1~\text{between \(x_0\) and \(x\) such that \(g'(c_1)=0\).}\\
		&g'(x_0)=0,g'(x)=0~\text{by Rolle's thoerem, there exists}~c_2~\text{between \(x_0\) and \(c\) such that \(g''(c_2)=0\).}\\
		&\vdots\\
		&g^{(n)}(x_0)=0,g^{(n)}(x)=0~\text{by Rolle's thoerem, there exists}~c_{n+1}~\text{between \(x_0\) and \(c_n\) such that \(g^{(n+1)}(c_{n+1})=0\).}
	\end{align*}
	Finally, set \(c:=c_{n+1}\) to conclude the proof.
\end{proof}
\begin{figure}[H]
	\centering
	\begin{tikzpicture}
		\draw[Latex-Latex] (-5,0)--(5,0);
		
		\node[] at (-4.75,0) {\([\)};
		
		\node[] at (4.75,0) {\(]\)};
		
		\node[below] at (-4.75,-0.15) {\(a\)};
		
		\node[below] at (4.75,-0.15) {\(b\)};
		
		\draw[fill] (-4,0) circle (1pt);
		
		\draw[fill] (4,0) circle (1pt);
		
		\node[below] at (-4,0) {\(x_0\)};
		
		\node[below] at (4,0) {\(x\)};
		
		\draw[fill] (2,0) circle (1pt);
		
		\node[below] at (2,0) {\(c_1\)};
		
		\draw[fill] (1,0) circle (1pt);
		
		\node[below] at (1,0) {\(c_2\)};
		
		\draw[fill] (0.5,0) circle (1pt);
		
		\node[below] at (0.5,0) {\(c_3\)};
		
		\draw[fill] (0,0) circle (1pt);
		
		\draw[fill] (-0.5,0) circle (1pt);
		
		\draw[fill] (-1,0) circle (1pt);
		
		\draw[fill] (-3.25,0) circle (1pt);
		
		\node[below] at (-3.25,0) {\(c_{n+1}\)};
	\end{tikzpicture}
	\caption{Visualization of the iterative process to find \(c_{n+1}\).}
\end{figure}

\begin{example}
	Why is Taylor's theorem so useful? We look at a few examples which illustrate this: set \(x_0=0\),
	\begin{enumerate}
		\item \(f\) is a polynomial of degree \(D\); \(P_n(t)\) will be the first terms of \(f\) up to degree \(n\).
		
		\item If \(f(t)=e^t\), we get 
		\begin{equation*}
			P_n(t)=\frac{1}{0!}+\frac{t}{1!}+\frac{t^2}{2!}+\frac{t^3}{3!}+\dots+\frac{t^n}{n!}.
		\end{equation*}
		
		\item If \(f(x)=\sin{x}\), we get 
		\begin{equation*}	
			P_n(t)=0+t+0-\frac{t^3}{3!}+0+\frac{t^5}{5!}+\dots
		\end{equation*}
	\end{enumerate}
\end{example}